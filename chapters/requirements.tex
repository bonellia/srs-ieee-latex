\chapter{Requirements}
% Specify the software system requirements to a level of detail sufficient for software design, development
% and verification of the software increment or release in process.
% The requirements should:
% a) be stated in conformance with all the characteristics described in 5.2 of this document;
% b) be cross-referenced to earlier versions or related documents;
% c) be uniquely identifiable;
% d) describe every input (stimulus) into the software system, every output (response) from the
% software system, and all functions performed by the software system in response to an input or in
% support of an output.



\section{Functions}
% Define the fundamental actions that have to take place in the software in accepting and processing the inputs and in processing and generating the outputs, including:
% a)    validity checks on the inputs;
% b)    exact sequence of operations;
% c)    responses to abnormal situations, including:
%       1)    overflow;
%       2)    communication facilities;
%       3)    hardware faults and failures; and
%       4)    error handling and recovery;
% d)    effect of parameters;
% e)    relationship of outputs to inputs, including:
%       1)    input/output sequences; and
%       2)    formulas for input to output conversion.
% It may be appropriate to partition the functional requirements into sub-functions or sub-processes. 
% This does not imply that the software design will also be partitioned that way.



\section{Performance requirements}
% Specify both the static and the dynamic numerical requirements placed on the software or on human
% interaction with the software as a whole.
% Static numerical requirements may include the following:
% a) the number of terminals to be supported;
% b) the number of simultaneous users to be supported; and
% c) the amount and type of information to be handled.
% Static numerical requirements are sometimes identified under a separate section entitled Capacity.
% Dynamic numerical requirements may include, for example, the numbers of transactions and tasks and
% the amount of data to be processed within certain time periods for both normal and peak workload
% conditions.
% The performance requirements should be stated in measurable terms.
% For example,
% 95 % of the transactions shall be processed in less than 1 s.
% rather than,
% An operator shall not have to wait for the transaction to complete.
% NOTE Numerical limits applied to one specific function are normally specified as part of the processing
% subparagraph description of that function.

\section{Usability requirements}
% Define usability and quality in use requirements and objectives for the software system that can include
% measurable effectiveness, efficiency, satisfaction criteria and avoidance of harm that could arise from
% use in specific contexts of use.
% NOTE Additional guidance on usability requirements can be found in ISO/IEC TR 25060.


\section{Interface requirements}
% Define all inputs into and outputs from the software system. The description should complement the
% interface descriptions in 9.6.4.1 through 9.6.4.5, and should not repeat information there.
% Each interface defined should include the following content:
% a) name of item;
% b) description of purpose;
% c) source of input or destination of output;
% d) valid range, accuracy and/or tolerance;
% e) units of measure;
% f) timing;
% g) relationships to other inputs/outputs;
% h) data formats;
% i) command formats; and
% j) data items or information included in the input and output.



\section{Logical database requirements}
% Specify the logical requirements for any information that is to be placed into a database, including:
% a)    types of information used by various functions;
% b)    frequency of use;
% c)    accessing capabilities;
% d)    data entities and their relationships;
% e)    integrity constraints;
% f)    security; and
% g)    data retention requirements



\section{Design constraints}
% Specify constraints on the system design imposed by external standards, regulatory requirements, or project
% limitations.

\section{Standards compliance}
% Specify the requirements derived from existing standards or regulations, including:
% a) report format;
% b) data naming;
% c) accounting procedures; and
% d) audit tracing.
% For example, this could specify the requirement for software to trace processing activity. Such traces
% are needed for some applications to meet minimum regulatory or financial standards. An audit trace
% requirement may, for example, state that all changes to a payroll database shall be recorded in a trace
% file with before and after values.

\section{Software System Attributes}
% Specify the required attributes of the software product. The following is a partial list of examples:
% a)  Reliability - Specify the factors required to establish the required reliability of the software system 
%     at time of delivery.
% b)  Availability - Specify the factors required to guarantee a defined availability level for the entire 
%     system such as checkpoint, recovery, and restart.
% c)  Security - Specify the requirements to protect the software from accidental or malicious access, 
%     use modification, destruction, or disclosure. Specific requirements in this area could include the 
%     need to:
%     1)    Utilize certain cryptographic techniques;
%     2)    Keep specific log or history data sets;
%     3)    Assign certain functions to different modules;
%     4)    Restrict communications between some areas of the program;
%     5)    Check data integrity for critical variables;
%     6)    Assure data privacy.
% d)  Maintainability - Specify attributes of software that relate to the ease of maintenance of the 
%     software itself. These may include requirements for certain modularity, interfaces, or complexity 
%     limitation. Requirements should not be placed here just because they are thought to be good design 
%     practices.
% e)  Portability - Specify attributes of software that relate to the ease of porting the software to other 
%     host machines and/or operating systems, including:
%     1)    Percentage of elements with host-dependent code;
%     2)    Percentage of code that is host dependent;
%     3)    Use of a proven portable language;
%     4)    Use of a particular compiler or language subset;
%     5)    Use of a particular operating system.



\section{Supporting Information}
% Additional supporting information to be considered includes:
% a) sample input/output formats, descriptions of cost analysis studies or results of user surveys;
% b) supporting or background information that can help the readers of the SRS;
% c) a description of the problems to be solved by the software; and
% d) special packaging instructions for the code and the media to meet security, export, initial loading
% or other requirements.
% The SRS should explicitly state whether or not these information items are to be considered part of the
% requirements.