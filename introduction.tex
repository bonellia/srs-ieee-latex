\chapter{Introduction}



\section{Purpose}
% Delineate the purpose of the software to be specified.

\section{Scope}
% Describe the scope of the software under consideration by
% a)   Identifying the software product(s) to be produced by name (e.g., Host DBMS, Report Generator, etc.);
% b)   Explaining what the software product(s) will do;
% c)   Describing the application of the software being specified, including relevant benefits, objectives, and 
%      goals;
% d)   Being consistent with similar statements in higher-level specifications (e.g., the system requirements 
%      specification), if they exist.


\section{Product Overview}

\subsection{Product Perspective}
% Define the system's relationship to other related products.
% If the product is an element of a larger system, then relate the requirements of that larger system to the 
% functionality of the product covered by the SRS.
% If the product is an element of a larger system, then identify the interfaces between the product covered by the 
% SRS and the larger system of which the product is an element.
% A block diagram showing the major elements of the larger system, interconnections, and external interfaces 
% can be helpful.
% Describe how the software operates within the following constraints:
%   a) System interfaces;
%   b) User interfaces;
%   c) Hardware interfaces;
%   d) Software interfaces
%   e) Communications interfaces;
%   f) Memory;
%   g) Operations;
%   h) Site adaptation requirements.


\subsubsection{System interfaces}
% List each system interface and identify the functionality of the software to accomplish the system requirement 
% and the interface description to match the system.


\subsubsection{User interfaces}
% Specify the following:
% a)   The logical characteristics of each interface between the software product and its users. This includes 
%      those configuration characteristics (e.g., required screen formats, page or window layouts, content of any 
%      reports or menus, or availability of programmable function keys) necessary to accomplish the software 
%      requirements.
% b)   All the aspects of optimizing the interface with the person who uses, maintains, or provides other support 
%      to the system. This may simply comprise a list of do's and don'ts on how the system will appear to the 
%      user. One example may be a requirement for the option of long or short error messages. This may also 
%      be specified in the Software System Attributes under a section titled Ease of Use.
% NOTE: A style guide for the user interface can provide consistent rules for organization, coding, and interaction of the
% user with the system.


\subsubsection{Hardware interfaces}
% Specify the logical characteristics of each interface between the software product and the hardware elements 
% of the system. This includes configuration characteristics (number of ports, instruction sets, etc.). It also covers 
% such matters as what devices are to be supported, how they are to be supported, and protocols. For example, 
% terminal support may specify full-screen support as opposed to line-by-line support.


\subsubsection{Software interfaces}
% Specify the use of other required software products (e.g., a data management system, an operating system, 
% or a mathematical package), and interfaces with other application systems (e.g., the linkage between an 
% accounts receivable system and a general ledger system).
% For each required software product, specify:
% a)   Name;
% b)   Mnemonic;
% c)   Specification number;
% d)   Version number;
% e)   Source.
% For each interface specify:
% a)   Discussion of the purpose of the interfacing software as related to this software product.
% b)   Definition of the interface in terms of message content and format. It is not necessary to detail any well- 
%      documented interface, but a reference to the document defining the interface is required.


\subsubsection{Communication interfaces}
% Specify the various interfaces to communications such as local network protocols.


\subsubsection{Memory constraints}
% Specify any applicable characteristics and limits on primary and secondary memory.


\subsubsection{Operations}
% Specify the normal and special operations required by the user such as
% a)   The various modes of operations in the user organization (e.g., user-initiated operations);
% b)   Periods of interactive operations and periods of unattended operations;
% c)   Data processing support functions;
% d)   Backup and recovery operations.
% NOTE: This is sometimes specified as part of the User Interfaces section.


\subsubsection{Site adaptation requirements}
% The site adaptation requirements include
% a)   Definition of the requirements for any data or initialization sequences that are specific to a given site,
%      mission, or operational mode (e.g., grid values, safety limits, etc.);
% b)   Specification of the site or mission-related features that should be modified to adapt the software to a
%      particular installation.


\subsection{Product functions}
% Provide a summary of the major functions that the software will perform. For example, an SRS for an
% accounting program may use this part to address customer account maintenance, customer statement, and
% invoice preparation without mentioning the vast amount of detail that each of those functions requires.

% Sometimes the function summary that is necessary for this part can be taken directly from the section of the
% higher-level specification (if one exists) that allocates particular functions to the software product.

% Note that for the sake of clarity
% a)   The product functions should be organized in a way that makes the list of functions understandable to the
%      acquirer or to anyone else reading the document for the first time.
% b)   Textual or graphical methods can be used to show the different functions and their relationships. Such a
%      diagram is not intended to show a design of a product, but simply shows the logical relationships among
%      variables.


\subsection{User characteristics}
% Describe those general characteristics of the intended groups of users of the product including characteristics
% that may influence usability, such as educational level, experience, disabilities, and technical expertise. This 
% description should not state specific requirements, but rather should state the reasons why certain specific
% requirements are later specified in specific requirements in subclause 9.5.9.
%
% NOTE: Where appropriate, the user characteristics of the SyRS and SRS should be consistent.


\subsection{Limitations}
% Provide a general description of any other items that will limit the supplier's options, including
% a)   Regulatory policies;
% b)   Hardware limitations (e.g., signal timing requirements);
% c)   Interfaces to other applications;
% d)   Parallel operation;
% e)   Audit functions;
% f)   Control functions;
% g)   Higher-order language requirements;
% h)   Signal handshake protocols (e.g., XON-XOFF, ACK-NACK);
% i)   Quality requirements (e.g., reliability)
% j)   Criticality of the application;
% k)   Safety and security considerations.
% l)   Physical/mental considerations


\section{Definitons}
% Provide definitions for any words or phrases that have special meaning beyond normal dictionary definitions.