\chapter{Requirements}
% Specify all of the software requirements to a level of detail sufficient to enable designers to design a software
% system to satisfy those requirements.
% Specify all of the software requirements to a level of detail sufficient to enable testers to test that the software
% system satisfies those requirements.
% At a minimum, describe every input (stimulus) into the software system, every output (response) from the software
% system, and all functions performed by the software system in response to an input or in support of an output.
% The specific requirements should:
% a)   Be stated in conformance with all the characteristics described in subclause 5.2 of this International
%      Standard.
% b)   Be cross-referenced to earlier documents that relate.
% c)   Be uniquely identifiable.


\section{Functions}
% Define all inputs into and outputs from the software system. The description should complement the interface
% descriptions in 2.3.1 through 2.3.5, and should not repeat information there.
% Each interface defined should include the following content:
% a)   Name of item;
% b)   Description of purpose;
% c)   Source of input or destination of output;
% d)   Valid range, accuracy, and/or tolerance;
% e)   Units of measure;
% f)   Timing;   
% g)   Relationships to other inputs/outputs;
% h)   Screen formats/organization; 
% i)   Window formats/organization;
% j)   Data formats;
% k)   Command formats;
% l)   Endmessages.


\section{Performance Requirements}
% Define the fundamental actions that have to take place in the software in accepting and processing the inputs
% and in processing and generating the outputs, including
% a)   Validity checks on the inputs
% b)   Exact sequence of operations
% c)   Responses to abnormal situations, including
%      1)   Overflow         
%      2)   Communication facilities         
%      3)   Error handling and recovery 
% d)   Effect of parameters 
% e)   Relationship of outputs to inputs, including 
%      1)   Input/output sequences         
%      2)   Formulas for input to output conversion
% It may be appropriate to partition the functional requirements into subfunctions or subprocesses. This does
% not imply that the software design will also be partitioned that way.


\section{Usability Requirements}
% Define usability (quality in use) requirements. Usability requirements and objectives for the software system
% include measurable effectiveness, efficiency, and satisfaction criteria in specific contexts of use.


\section{Interface Requirements}
% Specify both the static and the dynamic numerical requirements placed on the software or on human
% interaction with the software as a whole.
% Static numerical requirements may include the following:
% a)  The number of terminals to be supported;
% b)  The number of simultaneous users to be supported;
% c)  Amount and type of information to be handled.
% Static numerical requirements are sometimes identified under a separate section entitled Capacity.
% Dynamic numerical requirements may include, for example, the numbers of transactions and tasks and the
% amount of data to be processed within certain time periods for both normal and peak workload conditions.
% The performance requirements should be stated in measurable terms.
% For example,
%       95 % of the transactions shall be processed in less than 1 second.
% rather than,
%       An operator shall not have to wait for the transaction to complete.
% NOTE: Numerical limits applied to one specific function are normally specified as part of the processing subparagraph
% description of that function.
\section{Logical Database Requirements}
% Specify the logical requirements for any information that is to be placed into a database, including:
% a)   Types of information used by various functions;
% b)   Frequency of use;
% c)   Accessing   capabilities;
% d)   Data entities and their relationships;
% e)   Integrity   constraints;
% f)   Data retention requirements.


\section{Design Constraints}
% Specify constraints on the system design imposed by external standards, regulatory requirements, or project
% limitations.


\section{Software System Attributes}
% Specify the required attributes of the software product. The following is a partial list of examples:
% a)  Reliability - Specify the factors required to establish the required reliability of the software system at time
%     of delivery.
% b)  Availability - Specify the factors required to guarantee a defined availability level for the entire system
%     such as checkpoint, recovery, and restart.
% c)  Security - Specify the requirements to protect the software from accidental or malicious access, use
%     modification, destruction, or disclosure. Specific requirements in this area could include the need to:
%     1)   Utilize certain cryptographic techniques;
%     2)   Keep specific log or history data sets;
%     3)   Assign certain functions to different modules;
%     4)   Restrict communications between some areas of the program;
%     5)   Check data integrity for critical variables;
%     6)   Assure data privacy.
% d) Maintainability - Specify attributes of software that relate to the ease of maintenance of the software itself.
%    These may include requirements for certain modularity, interfaces, or complexity limitation. Requirements
%    should not be placed here just because they are thought to be good design practices.
% e)  Portability - Specify attributes of software that relate to the ease of porting the software to other host
%     machines and/or operating systems, including:
%     1) Percentage of elements with host-dependent code;
%     2) Percentage of code that is host dependent;
%     3) Use of a proven portable language;
%     4) Use of a particular compiler or language subset;
%     5) Use of a particular operating system.


\section{Supporting Information}
% The SRS should contain additional supporting information including
% a)   Sample input/output formats, descriptions of cost analysis studies, or results of user surveys;
% b)   Supporting or background information that can help the readers of the SRS;
% c)   A description of the problems to be solved by the software;
% d)   Special packaging instructions for the code and the media to meet security, export, initial loading, or other
%      requirements.
% The SRS should explicitly state whether or not these information items are to be considered part of the 
% requirements.