% In accordance with "Systems and software engineering — Life cycle processes — Requirements engineering"
% (ISO/IEC/IEEE29148) - https://standards.ieee.org/standard/29148-2011.html

\documentclass{scrreprt}
\usepackage{listings}
\usepackage{underscore}
\usepackage[bookmarks=true]{hyperref}
\usepackage[utf8]{inputenc}
\usepackage[english]{babel}
\hypersetup{
    pdftitle={Software Requirement Specification},
    pdfauthor={$<$Author$>$}, % author
    pdfsubject={TeX and LaTeX}, % subject of the document
    pdfkeywords={TeX, LaTeX, graphics, images, srs, software}, % list of keywords
    colorlinks=true, % false: boxed links; true: colored links
    linkcolor=blue, % color of internal links
    citecolor=black, % color of links to bibliography
    filecolor=black, % color of file links
    urlcolor=purple, % color of external links
    linktoc=page % only page is linked
}

\def\myversion{1.0 }
\date{}
\title{}
\usepackage{hyperref}
\begin{document}

\begin{flushright}
    \rule{14cm}{5pt}
    \vskip1cm
    {\bfseries
        \Huge{Software Requirements\\ Specification}\\
        \vspace{1.6cm}
        for\\
        \vspace{1.6cm}
        $<$Project Name$>$\\
        \vspace{1.6cm}
        \LARGE{Version \myversion approved}\\
        \vspace{1.6cm}
        Prepared by $<$Author$>$\\
        \vspace{1.6cm}
        $<$Company$>$\\
        \vspace{1.6cm}
        \today\\
    }
\end{flushright}

\tableofcontents

\chapter*{Revision History}

\begin{center}
    \begin{tabular}{|c|c|c|c|}
        \hline
	    Name & Date & Reason For Changes & Version\\
        \hline
	    $<$Rev Name 1$>$ & $<$Date 1$>$ & $<$Reason 1$>$ & $<$Ver 1$>$\\
        \hline
	    $<$Rev Name 2$>$ & $<$Date 2$>$ & $<$Reason 2$>$ & $<$Ver 2$>$\\
        \hline
    \end{tabular}
\end{center}

\chapter{Introduction}

\section{Purpose}
% Delineate the purpose of the software to be specified.

\section{Scope}
% Describe the scope of the software under consideration by
% a)   Identifying the software product(s) to be produced by name (e.g., Host DBMS, Report Generator, etc.);
% b)   Explaining what the software product(s) will do;
% c)   Describing the application of the software being specified, including relevant benefits, objectives, and 
%      goals;
% d)   Being consistent with similar statements in higher-level specifications (e.g., the system requirements 
%      specification), if they exist.


\section{Product Overview}

\subsection{Product Perspective}
% Define the system's relationship to other related products.
% If the product is an element of a larger system, then relate the requirements of that larger system to the 
% functionality of the product covered by the SRS.
% If the product is an element of a larger system, then identify the interfaces between the product covered by the 
% SRS and the larger system of which the product is an element.
% A block diagram showing the major elements of the larger system, interconnections, and external interfaces 
% can be helpful.
% Describe how the software operates within the following constraints:
%   a) System interfaces;
%   b) User interfaces;
%   c) Hardware interfaces;
%   d) Software interfaces
%   e) Communications interfaces;
%   f) Memory;
%   g) Operations;
%   h) Site adaptation requirements.


\subsubsection{System interfaces}
% List each system interface and identify the functionality of the software to accomplish the system requirement 
% and the interface description to match the system.


\subsubsection{User interfaces}
% Specify the following:
% a)   The logical characteristics of each interface between the software product and its users. This includes 
%      those configuration characteristics (e.g., required screen formats, page or window layouts, content of any 
%      reports or menus, or availability of programmable function keys) necessary to accomplish the software 
%      requirements.
% b)   All the aspects of optimizing the interface with the person who uses, maintains, or provides other support 
%      to the system. This may simply comprise a list of do's and don'ts on how the system will appear to the 
%      user. One example may be a requirement for the option of long or short error messages. This may also 
%      be specified in the Software System Attributes under a section titled Ease of Use.
% NOTE: A style guide for the user interface can provide consistent rules for organization, coding, and interaction of the
% user with the system.


\subsubsection{Hardware interfaces}
% Specify the logical characteristics of each interface between the software product and the hardware elements 
% of the system. This includes configuration characteristics (number of ports, instruction sets, etc.). It also covers 
% such matters as what devices are to be supported, how they are to be supported, and protocols. For example, 
% terminal support may specify full-screen support as opposed to line-by-line support.


\subsubsection{Software interfaces}
% Specify the use of other required software products (e.g., a data management system, an operating system, 
% or a mathematical package), and interfaces with other application systems (e.g., the linkage between an 
% accounts receivable system and a general ledger system).
% For each required software product, specify:
% a)   Name;
% b)   Mnemonic;
% c)   Specification number;
% d)   Version number;
% e)   Source.
% For each interface specify:
% a)   Discussion of the purpose of the interfacing software as related to this software product.
% b)   Definition of the interface in terms of message content and format. It is not necessary to detail any well- 
%      documented interface, but a reference to the document defining the interface is required.


\subsubsection{Communication interfaces}
% Specify the various interfaces to communications such as local network protocols.


\subsubsection{Memory constraints}
% Specify any applicable characteristics and limits on primary and secondary memory.


\subsubsection{Operations}
% Specify the normal and special operations required by the user such as
% a)   The various modes of operations in the user organization (e.g., user-initiated operations);
% b)   Periods of interactive operations and periods of unattended operations;
% c)   Data processing support functions;
% d)   Backup and recovery operations.
% NOTE: This is sometimes specified as part of the User Interfaces section.


\subsubsection{Site adaptation requirements}
% The site adaptation requirements include
% a)   Definition of the requirements for any data or initialization sequences that are specific to a given site,
%      mission, or operational mode (e.g., grid values, safety limits, etc.);
% b)   Specification of the site or mission-related features that should be modified to adapt the software to a
%      particular installation.


\subsection{Product functions}
% Provide a summary of the major functions that the software will perform. For example, an SRS for an
% accounting program may use this part to address customer account maintenance, customer statement, and
% invoice preparation without mentioning the vast amount of detail that each of those functions requires.

% Sometimes the function summary that is necessary for this part can be taken directly from the section of the
% higher-level specification (if one exists) that allocates particular functions to the software product.

% Note that for the sake of clarity
% a)   The product functions should be organized in a way that makes the list of functions understandable to the
%      acquirer or to anyone else reading the document for the first time.
% b)   Textual or graphical methods can be used to show the different functions and their relationships. Such a
%      diagram is not intended to show a design of a product, but simply shows the logical relationships among
%      variables.


\subsection{User characteristics}
% Describe those general characteristics of the intended groups of users of the product including characteristics
% that may influence usability, such as educational level, experience, disabilities, and technical expertise. This 
% description should not state specific requirements, but rather should state the reasons why certain specific
% requirements are later specified in specific requirements in subclause 9.5.9.
%
% NOTE: Where appropriate, the user characteristics of the SyRS and SRS should be consistent.


\subsection{Limitations}
% Provide a general description of any other items that will limit the supplier's options, including
% a)   Regulatory policies;
% b)   Hardware limitations (e.g., signal timing requirements);
% c)   Interfaces to other applications;
% d)   Parallel operation;
% e)   Audit functions;
% f)   Control functions;
% g)   Higher-order language requirements;
% h)   Signal handshake protocols (e.g., XON-XOFF, ACK-NACK);
% i)   Quality requirements (e.g., reliability)
% j)   Criticality of the application;
% k)   Safety and security considerations.
% l)   Physical/mental considerations


\section{Definitons}
% Provide definitions for any words or phrases that have special meaning beyond normal dictionary definitions.


\chapter{References}
% Include the following information regarding references:
% a)   Provide a complete list of all documents referenced elsewhere;
% b)   Identify each document by title, report number (if applicable), date and publishing organization; 
% c)   Specify the sources from which the references can be obtained.
% This information may be provided by reference to an appendix or to another document. The references
% information should be subdivided into a 'Compliance' section, containing references to those cited documents
% containing requirements which are included by that citation and a ‘Guidance' section, containing reference to
% those cited documents containing information, but no requirements.


\chapter{Specific requirements}
% Specify all of the software requirements to a level of detail sufficient to enable designers to design a software
% system to satisfy those requirements.
% Specify all of the software requirements to a level of detail sufficient to enable testers to test that the software
% system satisfies those requirements.
% At a minimum, describe every input (stimulus) into the software system, every output (response) from the software
% system, and all functions performed by the software system in response to an input or in support of an output.
% The specific requirements should:
% a)   Be stated in conformance with all the characteristics described in subclause 5.2 of this International
%      Standard.
% b)   Be cross-referenced to earlier documents that relate.
% c)   Be uniquely identifiable.


\section{External interfaces}
% Define all inputs into and outputs from the software system. The description should complement the interface
% descriptions in 2.3.1 through 2.3.5, and should not repeat information there.
% Each interface defined should include the following content:
% a)   Name of item;
% b)   Description of purpose;
% c)   Source of input or destination of output;
% d)   Valid range, accuracy, and/or tolerance;
% e)   Units of measure;
% f)   Timing;   
% g)   Relationships to other inputs/outputs;
% h)   Screen formats/organization; 
% i)   Window formats/organization;
% j)   Data formats;
% k)   Command formats;
% l)   Endmessages.


\section{Functions}
% Define the fundamental actions that have to take place in the software in accepting and processing the inputs
% and in processing and generating the outputs, including
% a)   Validity checks on the inputs
% b)   Exact sequence of operations
% c)   Responses to abnormal situations, including
%      1)   Overflow         
%      2)   Communication facilities         
%      3)   Error handling and recovery 
% d)   Effect of parameters 
% e)   Relationship of outputs to inputs, including 
%      1)   Input/output sequences         
%      2)   Formulas for input to output conversion
% It may be appropriate to partition the functional requirements into subfunctions or subprocesses. This does
% not imply that the software design will also be partitioned that way.


\section{Usability Requirements}
% Define usability (quality in use) requirements. Usability requirements and objectives for the software system
% include measurable effectiveness, efficiency, and satisfaction criteria in specific contexts of use.


\section{Performance Requirements}
% Specify both the static and the dynamic numerical requirements placed on the software or on human
% interaction with the software as a whole.
% Static numerical requirements may include the following:
% a)  The number of terminals to be supported;
% b)  The number of simultaneous users to be supported;
% c)  Amount and type of information to be handled.
% Static numerical requirements are sometimes identified under a separate section entitled Capacity.
% Dynamic numerical requirements may include, for example, the numbers of transactions and tasks and the
% amount of data to be processed within certain time periods for both normal and peak workload conditions.
% The performance requirements should be stated in measurable terms.
% For example,
%       95 % of the transactions shall be processed in less than 1 second.
% rather than,
%       An operator shall not have to wait for the transaction to complete.
% NOTE: Numerical limits applied to one specific function are normally specified as part of the processing subparagraph
% description of that function.
\section{Logical Database Requirements}
% Specify the logical requirements for any information that is to be placed into a database, including:
% a)   Types of information used by various functions;
% b)   Frequency of use;
% c)   Accessing   capabilities;
% d)   Data entities and their relationships;
% e)   Integrity   constraints;
% f)   Data retention requirements.


\section{Design Constraints}
% Specify constraints on the system design imposed by external standards, regulatory requirements, or project
% limitations.


\section{Software System Attributes}
% Specify the required attributes of the software product. The following is a partial list of examples:
% a)  Reliability - Specify the factors required to establish the required reliability of the software system at time
%     of delivery.
% b)  Availability - Specify the factors required to guarantee a defined availability level for the entire system
%     such as checkpoint, recovery, and restart.
% c)  Security - Specify the requirements to protect the software from accidental or malicious access, use
%     modification, destruction, or disclosure. Specific requirements in this area could include the need to:
%     1)   Utilize certain cryptographic techniques;
%     2)   Keep specific log or history data sets;
%     3)   Assign certain functions to different modules;
%     4)   Restrict communications between some areas of the program;
%     5)   Check data integrity for critical variables;
%     6)   Assure data privacy.
% d) Maintainability - Specify attributes of software that relate to the ease of maintenance of the software itself.
%    These may include requirements for certain modularity, interfaces, or complexity limitation. Requirements
%    should not be placed here just because they are thought to be good design practices.
% e)  Portability - Specify attributes of software that relate to the ease of porting the software to other host
%     machines and/or operating systems, including:
%     1) Percentage of elements with host-dependent code;
%     2) Percentage of code that is host dependent;
%     3) Use of a proven portable language;
%     4) Use of a particular compiler or language subset;
%     5) Use of a particular operating system.


\section{Supporting Information}
% The SRS should contain additional supporting information including
% a)   Sample input/output formats, descriptions of cost analysis studies, or results of user surveys;
% b)   Supporting or background information that can help the readers of the SRS;
% c)   A description of the problems to be solved by the software;
% d)   Special packaging instructions for the code and the media to meet security, export, initial loading, or other
%      requirements.
% The SRS should explicitly state whether or not these information items are to be considered part of the 
% requirements.


\chapter{Verification}
% Provide the verification approaches and methods planned to qualify the software. The information items for
% verification are recommended to be given in a parallel manner with the information items in subclause 4.1
% to 4.7.


\chapter{Appendices}
\section{Assumptions and dependencies}
% List each of the factors that affect the requirements stated in the SRS. These factors are not design
% constraints on the software but any changes to these factors can affect the requirements in the SRS. For
% example, an assumption may be that a specific operating system will be available on the hardware designated
% for the software product. If, in fact, the operating system is not available, the SRS would then have to change
% accordingly.


\section{Acronyms and abbreviations}
% Spell out or define all acronyms and abbreviations used in the documents.
% NOTE: This information may be provided by reference to one or more appendixes in the documents or by reference to
% other documents.


\end{document}
